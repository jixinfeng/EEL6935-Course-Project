\documentclass[conference]{IEEEtran}

\hyphenation{op-tical net-works semi-conduc-tor}

\begin{document}
\title{EEL6935 Course Project Proposal \\
    Sentence Classification by CNN}
\author{\IEEEauthorblockN{
    Caleb Bryant\IEEEauthorrefmark{1}, 
    Jixin Feng\IEEEauthorrefmark{2}, 
    and Hao Huang\IEEEauthorrefmark{1}}
\IEEEauthorblockA{Department of \\
    \IEEEauthorrefmark{1} Computer \& Information Science \& Engineering\\
    \IEEEauthorrefmark{2} Electrical \& Computer Engineering\\
    University of Florida,
    Gainesville, FL, 32611\\
    Email: \texttt{{\small\{cal2u,fengjixin,haohuang\}}@ufl.edu}}}

\maketitle

\begin{abstract}
    The volume of text -- unstructured text especially --  on the internet 
    has been generated with a increasingly drastic speed everyday. Unlike 
    human brain, traditional computer programs lacks the ability of extracting
    useful information from unstructured text with satisfactory procision, 
    not even trivial machine learning programs, while deep learning has shown 
    some promising result. In this paper, we propose to implement a sentense
    classification program based on convolutional neural networks (CNN) as 
    our course project for EEL6935 Big Data Ecosystem. 
\end{abstract}

\IEEEpeerreviewmaketitle

\section{Introduction}
    With tremendous volume of unstructured text generated everyday, on or off 
    internet, the demands for processing them and extracting useful information 
    from them has been constantly increasing. It is predicted that by 2020 the 
    total data volume of ``digital universe'' would be increased to 40 ZB 
    ($40\times 10^{21}$ bytes), which is about 50 times more than the size of 
    year 2010\cite{gantz2012digital}. Most of them would be generated by various 
    sources like news media, social network, medical record, etc. 
    
    Although those text data -- unstructured especially -- is a priceless source
    of knowledge and information, and can be easily comprehended by human, 
    but very limited amount of information can be extracted by computers running 
    traditional program. Hence an effective methods to process and analyze tremendous
    amount of unstructured text data is desperately needed.
    
    In this course project, we are aiming at a specific domain of text analysis
    job -- sentence classification. The goal of sentence classification is to assign
    proper pre-defined labels to a given sentence, so that the main subject of the 
    sentence of can be represented and then categorized\cite{allahyari2017brief}. 
    Mathematically, the classification model can be represented as:
    $$f:\mathcal{D}\rightarrow\mathcal{L}$$
    where $\mathcal{D}=\{d_0, d_1,\ldots, d_{n-1}\}$ is the set of sentences
    (sentence), and $\mathcal{L}=\{l_0, l_1,\ldots, l_{k-1}\}$ is the set of labels.
    Depending on whether multiple label is allow to assigned to a document, the 
    classification is called soft or hard\cite{gopal2010multilabel}.
    
    The performance of sentence classification can be evaluated with F-1 score?
    which can be defined as\cite{forman2003extensive}:
    $$F_1=\frac{2}{\frac{1}{r}+\frac{1}{p}}=\frac{2pr}{p+r}$$
    where $p=\frac{tpr}{tpr+fpr}$ stands for precision and $r=\frac{tpr}{tpr+fnr}$ 
    stands for recall.
    
    Historically, sentence classification had been done via methods based on 
    statistics and machine learning like naive Bayes, nearest neighbor, decision 
    tree, SVM, etc. In this proposal, we decide to implement sentence classifier
    using convolutional neural network (CNN), a deep learning model. By definition
    CNN uses multiple layers of convolving filters that apply on input data and 
    calculate the output after all layers of iteration. Although originally 
    build for computer vision, CNN has shown quite significant potentials in 
    natural language processing (NLP), especially in sentence classification
    \cite{kim2014convolutional}.

\section{Proposed System Model}

\section{Proposed Simulation Scenarios}

\section{Conclusion}
The conclusion goes here.

\bibliographystyle{IEEEtran}
\bibliography{eel6935_proposal}

\end{document}
